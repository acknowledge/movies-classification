\newpage
\section{Évaluation du modèle}

Dans cette partie nous avons testé l'algorithme avec les meilleurs paramètres trouvés auparavant et nous l'avons testé avec trois combination de poids pour la fonction de fitness (table \ref{table_param}):


\begin{table}[h]
\begin{tabular}{|l|l|l|l|}
\hline
\textbf{Criterion weigth} & \textbf{Exp1} & \textbf{Exp2} & \textbf{Exp3} \\ \hline
accuracy                 & 1             & 1             & 1             \\ \hline
Sensivity                & 0             & 1             & 1             \\ \hline
Specificity              & 0             & 0             & 1             \\ \hline
RMSE                     & 0             & 0             & 0.4           \\ \hline
Size                     & 0.1           & 0.1           & 0.1           \\ \hline
Fitness max              & 0.732         & 0.81          & 0.748         \\ \hline
Average                  & 0.579         & 0.65          & 0.58          \\ \hline
accuracy                 & 0.734         & 0.725         & 0.75          \\ \hline
Sensivity                & 0.657         & 0.908         & 0.676         \\ \hline
Specificity              & 0.8           & 0.571         & 0.820         \\ \hline
\end{tabular}
\label{table_param}
\end{table}\\
\\
On peut voir du tableau ci-dessous que le la valeur maximale du "fitness" est parvenu dans l'expérience 2 ou on donne un poids assez grand a le \textbf{taux d'erreur (accuracy) et la sensivité} et un poids plus petit à la spécificité.
\\\\
En plus que la valeur de Fitness, on peut voir aussi les règles générées par le système.
Les règles du système de l'expérience 2 sont les suivantes:
\begin{enumerate}
\item  IF V5 is MF 1 AND V4 is MF 0 THEN Out is MF 1
\item  IF V5 is MF 1 AND V4 is MF 0 THEN Out is MF 1
\item  IF V15 is MF 1 AND V13 is MF 1 THEN Out is MF 1
\item  IF V9 is MF 1 AND V2 is MF 0 AND V4 is MF 0 AND V8 is MF 1 THEN Out is MF 1
\item IF V9 is MF 1 AND V2 is MF 0 AND V4 is MF 0 AND V8 is MF 1 THEN Out is MF 1
\item IF V8 is MF 0 AND V9 is MF 1 THEN Out is MF 0
\item IF V8 is MF 0 AND V9 is MF 1 THEN Out is MF 0
\item IF V10 is MF 1 AND V11 is MF 0 AND V15 is MF 0 AND V13 is MF 1 THEN Out is MF 1 ELSE : Out is 1  
\end{enumerate}\\
\\
Chaque variable VX correspond à une caractéristique utilisée pour générer le modèle. Ces caractéristiques sont expliquées sur la page \textbf{\url{http://archive.ics.uci.edu/ml/datasets/Arrhythmia}}.
Si on prend par exemple la règle 1 on peut la décrire dans la façon suivante: si V5 (QRS) est "grand"  et v4 (Weight) est petit alors le patient à bonne chance d'avoir troubles du rythme cardiaque (sortie MF 1).

On peut voir que la règle 1-2,3,6-7 sont assai simple et facilement compréhensible. 
Les règles 6-7 expliquent, par exemple, que si le "T-interval"(V8) est petit et le P-interval (v9) est grand, alors le patient a bon chance de n'avoir par de trubles du rythme cardiaque.


 
 
 
 
 
 

