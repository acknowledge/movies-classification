\newpage
\section*{Évaluation du modèle}

Dans cette partie nous avons testé l'algorithme avec les meilleurs paramètres trouvés auparavant et nous l'avons testé avec trois combinaisons de poids pour la fonction de fitness (table \ref{table_param}) :


\begin{table}[h]
  \centering
  \begin{tabular}{|l|l|l|l|}
  \hline
  \textbf{Criterion weight} & \textbf{Exp1} & \textbf{Exp2} & \textbf{Exp3} \\ \hline
  accuracy                 & 1             & 1             & 1             \\ \hline
  Sensivity                & 0             & 1             & 1             \\ \hline
  Specificity              & 0             & 0             & 1             \\ \hline
  RMSE                     & 0             & 0             & 0.4           \\ \hline
  Size                     & 0.1           & 0.1           & 0.1           \\ \hline
  Fitness max              & 0.732         & 0.81          & 0.748         \\ \hline
  Average                  & 0.579         & 0.65          & 0.58          \\ \hline
  accuracy                 & 0.734         & 0.725         & 0.75          \\ \hline
  Sensivity                & 0.657         & 0.908         & 0.676         \\ \hline
  Specificity              & 0.8           & 0.571         & 0.820         \\ \hline
  \end{tabular}
  \caption{\label{table_param} Table de résultats}
\end{table}

On peut voir du tableau ci-dessus que la valeur maximale du "fitness" est arrivée dans l'expérience 2 où on donne un poids assez grand au \textbf{taux d'erreur (accuracy) et à la sensivité} et un poids plus petit à la spécificité.


En plus de la valeur de fitness, on peut aussi voir les règles générées par le système. Les règles du système de l'expérience 2 sont les suivantes:

\begin{enumerate}
  \item IF V5 is MF 1 AND V4 is MF 0 THEN Out is MF 1
  \item IF V5 is MF 1 AND V4 is MF 0 THEN Out is MF 1
  \item IF V15 is MF 1 AND V13 is MF 1 THEN Out is MF 1
  \item IF V9 is MF 1 AND V2 is MF 0 AND V4 is MF 0 AND V8 is MF 1 THEN Out is MF 1
  \item IF V9 is MF 1 AND V2 is MF 0 AND V4 is MF 0 AND V8 is MF 1 THEN Out is MF 1
  \item IF V8 is MF 0 AND V9 is MF 1 THEN Out is MF 0
  \item IF V8 is MF 0 AND V9 is MF 1 THEN Out is MF 0
  \item IF V10 is MF 1 AND V11 is MF 0 AND V15 is MF 0 AND V13 is MF 1 THEN Out is MF 1 ELSE : Out is 1  
\end{enumerate}


Chaque variable VX correspond à une caractéristique utilisée pour générer le modèle. Ces caractéristiques sont expliquées sur la page \url{http://archive.ics.uci.edu/ml/datasets/Arrhythmia}.
Si on prend par exemple la règle 1, on peut la décrire de la façon suivante : si V5 (QRS) est "grand" et V4 (Weight) est petit, alors le patient a de bonnes chances d'avoir des troubles d'arythmie cardiaque (sortie MF 1).

On peut voir que les règles 1-2,3,6-7 sont également simple et facilement compréhensible. 
Les règles 6-7 expliquent, par exemple, que si le "T-interval" (V8) est petit et le P-interval (V9) est grand, alors le patient a de bonnes chances de n'avoir par de troubles d' arythmie cardiaque.


 
 
 
 
 
 

